\documentclass[10pt,twocolumn,prl,aps,floatfix,superscriptaddress,longbibliography]{revtex4-1}

\usepackage{hyperref}

% for pretty printing LaTeX code
\usepackage{listings}
\usepackage{xcolor}
\usepackage{listings}
\lstset
{
    language=[LaTeX]TeX,
    breaklines=true,
    basicstyle=\tt\scriptsize,
    keywordstyle=\color[HTML]{0080ff},
    identifierstyle=\color[HTML]{87a96b},
    commentstyle=\color{gray}, 
    stringstyle=\ttfamily\color{red},
    numbers=left,
    numbersep=1.0ex,
    numberstyle=\tiny\color{gray},
}

\usepackage{pdfpages} % include pdfs
\usepackage{pgffor} % for loops
\usepackage{xr} % referencing the supplement

% Fix for a pdfpages rotation bug with revtex
\makeatletter
\AtBeginDocument{\let\LS@rot\@undefined}
\makeatother

% the name of the supplement PDF file
\def\supplementfilename{supplement}

% define external document for referencing
% supp: prefix for equations and figures when referencing
\externaldocument[supp:]{\supplementfilename}

% Determine the number of pages 
% in the supplement file and store
\pdfximage{\supplementfilename.pdf}
\def\numbersupplementpages{\the\pdflastximagepages}

% Are we submitting to the arXiv? 
% Un-comment the appropriate line
\newif\ifarXiv
\arXivtrue 
% \arXivfalse

\begin{document}

\title{Combining supplemental material into your manuscript as a separate pdf
file suitable for posting to the arXiv}

\author{Adrian Del Maestro}
\email{Adrian.DelMaestro@uvm.edu}
\affiliation{Department of Physics, University of Vermont, Burlington, VT 05405, USA}
\affiliation{Institut f\"ur Theoretische Physik, Universit\"at Leipzig, D-04103, Leipzig, Germany}

\begin{abstract}
    Here I describe a convenient method for managing manuscript and supplemental
    material files separately, then easily combining them for submission to
    journals or the arXiv while using a common bibtex bibliography file.
\end{abstract}
\maketitle

% ---------------------------------------------------------------------------------
% Introduction
% ---------------------------------------------------------------------------------
Journals often require that supplemental material be uploaded as a
self-contained \texttt{.pdf} file while an arXiv submission needs all material to be
included in a single \texttt{.tex} file.  This can cause problems when using
bibtex, especially if the same references are cited in both the manuscript and supplement.
Here I present a solution to this problem based on an answer posted on
\LaTeX~Stack Exchange \cite{tex} that I have recently used in two arXiv
postings \cite{Sengupta:2018pv, Barghathi:2018cu}.

The header of the main manuscript file should include the following packages, code and definitions:
%
\begin{lstlisting}
\usepackage{pdfpages} % include pdfs
\usepackage{pgffor} % for loops
\usepackage{xr} % referencing the supplement

% Fix for a pdfpages rotation bug with revtex
\makeatletter
\AtBeginDocument{\let\LS@rot\@undefined}
\makeatother

% the name of the supplement PDF file
\def\supplementfilename{supplement}

% define external document for referencing
% supp: prefix for referencing eqs. and figs.
\externaldocument[supp:]{\supplementfilename}

% Determine the number of pages 
% in the supplement file and store
\pdfximage{\supplementfilename.pdf}
\def\numbersupplementpages{\the\pdflastximagepages}

% Are we submitting to the arXiv? 
% Un-comment the appropriate line
\newif\ifarXiv
\arXivtrue 
% \arXivfalse
\end{lstlisting}
%
where \texttt{pdfpages} facilitates the inclusion of external multi-page PDF files in \LaTeX~documents \cite{pdfpages}, \texttt{pgffor} allows one to write simple for loops and \texttt{xr} allows for inter-document referencing.  The bug fix for dealing with page rotation on lines 4-7 was found here \cite{rotate}. For convenience we define two variables \verb!\supplementfilename! and \verb!\numbersupplementpages! to store the name and number of pages of the supplement file pdf (\texttt{supplement.pdf} here).  We compute the number of pages on line 20. Line 24 defines a new variable which we use to control whether or not to include the supplement which we set to true in line 25.  One can comment line 25 and uncomment line 26 if the supplement does not need to be included (\emph{i.e.}~for journal submission).

After generating the appropriate \texttt{supplement.pdf} with \verb!\numbersupplementpages! pages using a separate latex file, one can then include it at the end of the manuscript (after \texttt{\textbackslash bibliography\{refs\}} but before \texttt{\textbackslash end\{document\}}) via the following loop: \\ 
%
\begin{lstlisting}
\ifarXiv
    \foreach \x in {1,...,\numbersupplementpages}
    {
        \clearpage
        \includepdf[pages={\x,{}}]{\supplementfilename.pdf}
    }
\fi
\end{lstlisting}
%
See the included \texttt{supplement.tex} file for details on how to add unique
labels to equations and references.

The arXiv submission should include \texttt{manuscript.tex},
(\texttt{manuscript.bbl} if using \texttt{bibtex}) and
\texttt{supplement.pdf} while for the journal submission, you can simply
set \verb!\arXivfalse! in the header.

Document elements (equations, sections, figures, etc.) in the supplement can be referenced in the main document by prepending a \texttt{supp:} to the label, i.e. \verb!\ref{supp:eq:abc}! refers to Eq.~(\ref{supp:eq:abc}). If we include an equation in the main text:
%
\begin{equation}
    J_n(x) = \frac{1}{\pi}\int_0^\pi \cos (n \tau - x \sin \tau) d\tau
\label{eq:bessel}
\end{equation}
%
it can also be uniquely referenced back from the supplement via \verb!\ref{main:eq:bessel}!.


\bibliography{refs}

\ifarXiv
    \foreach \x in {1,...,\numbersupplementpages}
    {
        \clearpage
        \includepdf[pages={\x,{}}]{\supplementfilename.pdf}
    }
\fi


\end{document}
